\documentclass[11pt]{article}
\usepackage{listings}
\usepackage{color}

\definecolor{mygreen}{rgb}{0,0.6,0}
\definecolor{mygray}{rgb}{0.5,0.5,0.5}
\definecolor{mymauve}{rgb}{0.58,0,0.82}

\lstset{
    frame=single,
    language=Python,
    backgroundcolor=\color{white},
    basicstyle=\footnotesize,
    breaklines=true,
    commentstyle=\color{mygreen},
    keywordstyle=\color{blue},
    stringstyle=\color{mymauve},
    stepnumber=1,
    showstringspaces=false,
    tabsize=1,
    breakatwhitespace=false,
    literate=
        {á}{{\'a}}1
        {à}{{\`a}}1
        {ã}{{\~a}}1
        {é}{{\'e}}1
        {ê}{{\^e}}1
        {í}{{\'i}}1
        {ó}{{\'o}}1
        {õ}{{\~o}}1
        {ú}{{\'u}}1
        {ü}{{\"u}}1
        {ç}{{\c{c}}}1
}

\title{Lista de exercícios - PCS3438}
\author{
    Gustavo Freitas de Sá Oliveira - 11261062
    \and
    Roberta Boaventura Andrade - 11260832
}
\date{05/12/2022}

\begin{document}

\maketitle

\section{Introdução}

Os programas desenvolvidos para a resolução dos problemas descritos foram desenvolvidos em Python. Foram utilizadas as bibliotecas: pandas (leitura e tratamento de dados), numpy (operações matemáticas) e sklearn (modelos de inteligência artificial e métodos de validação).

\section{Questão 1}

Considerando os dados presentes no arquivo class01.csv, treine o algoritmo Naive Bayes Gaussiano utilizando a metodologia de validação cruzada holdout (utilize para treino as 350 primeiras linhas e para validação as demais). Qual o valor da acurácia a base de treino? Qual o valor da acurácia na base de validação?

\lstinputlisting[firstline = 9, lastline = 20]{../src/q1.py}

O código apresentado acima lê a base de dados e inicializa um objeto \emph{GaussianNB}, que recebe os dados como parâmetros nas funções \emph{fit} e \emph{score}. Essas funções são responsáveis por treinar e avaliar o algoritmo, respectivamente.

O método de validação hold-out é implementado com a função \emph{train\_test\_split}, que separa a base em duas porções para treino e validação.

Saída observada:

\begin{lstlisting}
    Acurácia na base de treino: 0.76
    Acurácia na base de validação: 0.6276923076923077
\end{lstlisting}

Nesse exemplo, percebe-se que a acurácia na base de treino foi maior do na base de validação. Esse fenômeno pode estar relacionado com o \emph{overfitting}, refletindo uma "memorização" dos dados pelo modelo.
\section{Questão 2}

Considerando os dados presentes no arquivo class02.csv, treine o algoritmo 10-Nearest Neighbors (KNN com \emph{k} = 10 e distância Euclidiana), utilizando a metodologia de validação cruzada k-fold com 5 folds. Considere que a primeira pasta de validação seja formada pelas primeiras 20\% linhas do arquivo, que a segunda pasta de validação seja formada pelas 20\% linhas seguintes, e assim por diante, até atingir a última pasta, formada pelas 20\% linhas finais da base. Qual a acurácia média para a base de validação?

\lstinputlisting[firstline = 9, lastline = 28]{../src/q2.py}

O código apresentado acima lê a base de dados e inicializa um objeto \emph{KNeighborsClassifier}, que recebe os dados como parâmetros nas funções \emph{fit} e \emph{score}. Essas funções são responsáveis por treinar e avaliar o algoritmo, respectivamente.

O método de validação k-fold é implementado com o objeto \emph{Kfold}, com parâmetro 5. Esse método divide a base em partes iguais, utilizando uma de cada vez para validação, e as demais para treino.

Saída observada:

\begin{lstlisting}
    Acurácia na base de treino: 0.8661666666666668
    Acurácia na base de validação: 0.8386666666666667
\end{lstlisting}

Assim como no exemplo anterior, percebe-se uma maior acurácia na base de treino. Isso pode estar também relacionado com o fenômeno anteriormente descrito.
\section{Questão 3}

Considerando os dados presentes no arquivo reg01.csv, obtenha um modelo de regressão linear com regularização L1 (LASSO com $\alpha$\ = 1) utilizando a metodologia Leave-One-out. Qual o valor médio do Root Mean Squared Error (RMSE) para a base de treino e para a base de validação?

\lstinputlisting[firstline = 9, lastline = 29]{../src/q3.py}

O código apresentado acima lê a base de dados e inicializa um objeto \emph{Lasso}, que recebe os dados como parâmetros nas funções \emph{fit} e \emph{predict}. Essas funções são responsáveis por treinar e avaliar o algoritmo, respectivamente.

O método de validação Leave-One-Out é implementado com o objeto \emph{LeaveOneOut}. Esse método é semelhante ao k-fold anteriormente exemplificado, mas aqui o número de folds é igual ao número de exemplos na base de dados.

A obtenção do RMSE é feita retirando a raiz quadrada do erro médio quadrado das predições do modelo.

Saída observada:

\begin{lstlisting}
    RMSE na base de treino: 19.220259837710355
    RMSE na base de validação: 15.465218791702428
\end{lstlisting}

O \emph{RMSE} foi maior na base de validação que na base de treino que na base de validação. A escolha do método de validação (Leave-One-Out) pode ter influenciado nesse fenômeno, uma vez que, nesse método, as validações acontecem para cada linha da base de dados.
\section{Questão 4}

Considerando os dados do arquivo reg02.csv, treine árvores de regressão, sem realizar podas, utilizando a metodologia de validação cruzada k-fold com \emph{k} = 5. Qual o valor do Mean Absolute Error (MAE) para a base de treino? Qual o valor médio do MAE para a base de validação?

\lstinputlisting[firstline = 9, lastline = 29]{../src/q4.py}

O código apresentado acima lê a base de dados e inicializa um objeto \emph{DecisionTreeRegressor}, que recebe os dados como parâmetros nas funções \emph{fit} e \emph{predict}. Essas funções são responsáveis por treinar e avaliar o algoritmo, respectivamente.

O método de validação k-fold é o mesmo mostrado anteriormente.

A obtenção do MAE é feita retirando o erro absoluto médio das predições do modelo.

Saída observada:

\begin{lstlisting}
    MAE na base de treino: 0.0
    MAE na base de validação: 43.22051929803169
\end{lstlisting}

Nesse exemplo, o \emph{MAE} foi nulo na base de treino. Esse resultado é esperado em virtude do tipo de modelo escolhido (árvore de regressão). O \emph{MAE} na base de validação, entretanto, foi diferente de zero.

\end{document}